\section{syndat program}

{\bf syndat} program makes synthetic seismograms by convolution
of Green functions ({\bf green} program output) with the
seismic moment tensor of the chosen event.

\subsection {Command line}
                                                                               
\noindent {\bf syndat}

\noindent {\bf syndat} $<$ {\it parameter\_file}

\noindent {\bf syndat} $<<$ WORD \\
\noindent ........  \\
\noindent ........  \\
\noindent WORD

\noindent There are three different ways to start the {\bf syndat} program:
\begin{itemize}
\item interactive dialog for setting input parameters;
\item single shell command with redirection of the standard input from a parameter
file {\it parameter\_file} containing exactly the same information and
in the same order as in the interactive dialog - one answer per line;
\item direct shell script. In this case, the contents of the parameter file are
directly included in the shell script between the delimiters ``WORD". See examples.
\end{itemize}
\noindent The parameter file consists of four lines:
\begin{description}
\item[Line 1:] {\it cmt\_event}\\
  This is the path to the file with the CMT solution for a single event. It must be the same
  {\it cmt\_event} file as used in the construction of the Green functions by the
 {\bf green} program. It uses part of the file
  defining the seismic moment tensor components, scalar moment, and focal planes. \\
  {\bf Format:} Any string up to 256 characters long.
\item[Line 2:] {\it in\_dbname } \\
{\it in\_dbname } is the input database name for the input Green functions. The {\it .wfdisc} relation of the
  {\it in\_dbname } database must be the output {\it .wfdisc} relation of the
  {\bf green} program. \\
  {\bf Format:} Any string up to 256 characters long. 
\item[Line 3:] {\it plane} \\
  This defines the method for introducing the moment tensor components.
  If $plane=0$, the tensor components come in directly from the {\it cmt\_event}
  file as is. If $plane=1$, the program takes the scalar moment and angles from
  focal plane 1 and computes the moment tensor components. For $plane=2$,
  the program computes the tensor components using the scalar moment and
  focal plane 2. \\
  {\bf Format:} Unformatted.
\item[Line 4:] {\it out\_dbname} \\
  This is the output database name for the {\it .wfdisc} relation for 
  synthetic seismograms. Each row of the {\it .wfdisc} relation refers to 
  a binary file with synthetic data in nm/sec. \\
  {\bf Format:} Any string up to 256 characters long.
\item[Line 5:] {\it datatype} \\
  This defines the type of synthetic data. If $datatype=0$ the 
  output synthetic waveforms are accelerograms in $nm/s^2$. This is 
  recommended native {\bf Mineos} output. If $datatype=1$ the
  output is velocity waveform in $nm/s$, and if $datatype=2$ the
  output is displacement in $nm$. This is done by additional
  conversation of accelerograms to velocity or displacement.
\end{description}
%
% Input data
%
\subsection {Input data}

\textbf{\large \emph{cmt\_event.}} See description in section 5.2. \\
\textbf{\large \emph{in\_dbname.}} {\it out\_dbname} database for the {\it .wfdisc}
relation from the  {\bf green} program. 
\subsection {Output data}
\textbf{\large \emph{out\_dbname.}} Database name for the final synthetic seismograms
referenced by the {\it .wfdisc} relation.

